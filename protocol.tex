% ****** Start of file aipsamp.tex ******
%
%   This file is part of the AIP files in the AIP distribution for REVTeX 4.
%   Version 4.1 of REVTeX, October 2009
%
%   Copyright (c) 2009 American Institute of Physics.
%
%   See the AIP README file for restrictions and more information.
%
% TeX'ing this file requires that you have AMS-LaTeX 2.0 installed
% as well as the rest of the prerequisites for REVTeX 4.1
%
% It also requires running BibTeX. The commands are as follows:
%
%  1)  latex  aipsamp
%  2)  bibtex aipsamp
%  3)  latex  aipsamp
%  4)  latex  aipsamp
%
% Use this file as a source of example code for your aip document.
% Use the file aiptemplate.tex as a template for your document.
\documentclass[%
 jcp, numerical,
 amsmath,amssymb,
%preprint,%
 preprint,%
%author-year,%
%author-numerical,%
]{revtex4-1}
\usepackage{graphicx}% Include figure files
\usepackage{dcolumn}% Align table columns on decimal point
\usepackage{bm}% bold math
\usepackage{color} \newcommand{\cor}{\color{red}}\newcommand{\cog}{\color{green}}\newcommand{\com}{\color{blue}}
\usepackage{tikz}
\usetikzlibrary{trees}
\begin{document}

\title{AZP group project data storage protocol}
\author{Andrew P. Santos} 
\affiliation{Department of Chemical and Biological Engineering, Princeton University, Princeton, NJ 08544, USA}
\date{\today}
\maketitle
%\tableofcontents
\section{Data Structure}
\label{sec:structure}
\tikzstyle{every node}=[draw=black,thick,anchor=west]
\tikzstyle{selected}=[draw=red,fill=red!30]
\tikzstyle{optional}=[dashed,fill=gray!50]
\begin{tikzpicture}[%
  grow via three points={one child at (0.5,-0.7) and
  two children at (0.5,-0.7) and (0.5,-1.4)},
  edge from parent path={(\tikzparentnode.south) |- (\tikzchildnode.west)}]
  \node {project-name}
    child { node [optional] {README.project-name (Project and its data structure overview)}}
    child { node {publication}    
    	child { node [optional] {README.publication}}
	    child { node [optional] {Tex files (*tex, *bib, publisher format files)}}
    	child { node {figures}
    	    child { node [optional] {README.figures (Explanation of how to generate figures)}}
    		child { node [optional] {figures}}
    	}				
	    child [missing] {}				
	    child [missing] {}	
    }		
    child [missing] {}				
    child [missing] {}	
    child [missing] {}				
    child [missing] {}				
    child [missing] {}	
    child { node [selected] {code}
       	child { node {pre-processing}
    	    child { node [optional] {README.pre (pre-processing scripts instructions)}}
       		child { node [optional] {code files}}
       	}				
	    child [missing] {}				
	    child [missing] {}	
       	child { node {simulation}
    	    child { node [optional] {README.simulation (Compilation instructions)}}
       		child { node [optional] {code files}}
       	}				
	    child [missing] {}				
	    child [missing] {}
       	child { node {post-processing}
    	    child { node [optional] {README.post (post-processing scripts instructions)}}
       		child { node [optional] {code files}}
       	}				
	    child [missing] {}				
	    child [missing] {}	
    }		
    child [missing] {}				
    child [missing] {}	
    child [missing] {}				
    child [missing] {}				
    child [missing] {}	
    child [missing] {}    	
    child [missing] {}
    child [missing] {}    	
    child [missing] {}
;
\end{tikzpicture}
\begin{tikzpicture}[%
  grow via three points={one child at (0.5,-0.7) and
  two children at (0.5,-0.7) and (0.5,-1.4)},
  edge from parent path={(\tikzparentnode.south) |- (\tikzchildnode.west)}]
  \node {project-name (continued)}
    child { node {Data}
        child { node [optional] {README.data}
        	child { node [optional] {List the types of models}}
	        child { node [optional] {List the conditions run}}
   	        child { node [optional] {List the simulation files and how to run the simulation}}
 	        child { node [optional] {List the post-processing files and how to run the scripts}}
        }
		child [missing] {}				
		child [missing] {}	
		child [missing] {}				
		child [missing] {}			
       	child { node {molecular-parameters}
    	    child { node {molecule-name1}
    	    	 child { node [optional] {Readme.molecule-name1 (explanation of molecule parameter files)}}
    	    	 child { node [optional] {parameter files}}
    	    }		
	 	    child [missing] {}
	 	    child [missing] {}
    	    child { node {molecule-name2} }
     	    child { node {molecule-name3} }
      	    child { node {...} }
       	}				
	    child [missing] {}				
	    child [missing] {}
	    child [missing] {}				
	    child [missing] {}		   
	    child [missing] {}	   
 	    child [missing] {}
       	child { node {condition-1}
   	    	child { node {simulation}
      	    	 child { node [optional] {simulation submission files}}
      	    	 child { node [optional] {simulation input files}}
      	    	 child { node [optional] {simulation output files}}      	    	 
	        }
	 	    child [missing] {}				
	 	    child [missing] {}
	 	    child [missing] {}
   	    	child { node {post-processing}
      	    	 child { node [optional] {post-processing submission files}}
      	    	 child { node [optional] {post-processing input files}}
      	    	 child { node [optional] {post-processing output files}}      	    	 
	        }
	 	    child [missing] {}				
	 	    child [missing] {}
	 	    child [missing] {}  
		}
	    child [missing] {}
 	    child [missing] {}
	    child [missing] {}				
	    child [missing] {}
	    child [missing] {}				
	    child [missing] {}		   
	    child [missing] {} 	  
   	    child { node {condition-2} }  
   	    child { node {condition-3} }  
  	    child { node {...} }
    }
;
\end{tikzpicture}
	\bibliography{library.bib}
\end{document}
